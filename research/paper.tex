\documentclass[12pt,a4paper]{article}

% Packages
\usepackage[utf8]{inputenc}
\usepackage[english]{babel}
\usepackage{amsmath,amssymb,amsthm}
\usepackage{graphicx}
\usepackage{hyperref}
\usepackage{geometry}
\usepackage{natbib}
\usepackage{algorithm}
\usepackage{algpseudocode}
\usepackage{listings}
\usepackage{xcolor}

\geometry{margin=1in}

% Code styling
\lstset{
    language=Python,
    basicstyle=\ttfamily\small,
    keywordstyle=\color{blue},
    commentstyle=\color{gray},
    stringstyle=\color{red},
    showstringspaces=false,
    breaklines=true,
    frame=single,
    numbers=left,
    numberstyle=\tiny\color{gray}
}

% Title and Author
\title{Advanced Numerical Analysis of Non-Newtonian Glacier Flow: \\
\large A Comprehensive Study of Glen's Law Implementation}

\author{Your Name \\
\small Your Institution \\
\small \texttt{your.email@institution.edu}}

\date{\today}

\begin{document}

\maketitle

\begin{abstract}
This paper presents a comprehensive numerical analysis of non-Newtonian glacier flow using the Finite Element Method (FEM). Building upon Glen's flow law, we develop and validate advanced computational approaches for modeling ice dynamics in alpine glaciers. Through rigorous mathematical formulation and numerical experimentation, we investigate the behavior of ice flow under varying rheological parameters, boundary conditions, and geometric configurations. Our results demonstrate the critical importance of non-linear viscosity models in accurately capturing glacier dynamics and provide insights into the computational challenges inherent in solving these highly non-linear systems.

\textbf{Keywords:} Glacier dynamics, Finite Element Method, Non-Newtonian flow, Glen's law, Computational geophysics
\end{abstract}

\section{Introduction}

Glacier dynamics play a crucial role in understanding climate change, sea-level rise, and alpine hazards. The flow of ice, governed by complex non-Newtonian rheology, presents significant challenges for computational modeling \citep{reference1}.

\subsection{Motivation}

% TODO: Add motivation for your research

\subsection{Objectives}

The primary objectives of this research are:
\begin{enumerate}
    \item To develop a robust FEM implementation for non-Newtonian glacier flow
    \item To investigate the influence of Glen's flow law parameters on ice dynamics
    \item To validate numerical results against analytical and experimental data
    \item To analyze the computational efficiency of various solution approaches
\end{enumerate}

\subsection{Contributions}

% TODO: Describe your novel contributions

\section{Mathematical Formulation}

\subsection{Governing Equations}

The governing equations for glacier flow are derived from the Stokes equations for incompressible flow, coupled with Glen's flow law for ice rheology.

% TODO: Add detailed mathematical formulation

\subsection{Glen's Flow Law}

Glen's flow law describes the non-Newtonian behavior of ice:
\begin{equation}
    \dot{\epsilon}_{ij} = A \tau^{n-1} \tau_{ij}
\end{equation}

where $\dot{\epsilon}_{ij}$ is the strain rate tensor, $\tau$ is the effective stress, $A$ is the flow parameter, and $n$ is the exponent (typically $n = 3$ for ice).

% TODO: Expand on the mathematical framework

\section{Numerical Methods}

\subsection{Finite Element Discretization}

% TODO: Describe your FEM implementation

\subsection{Non-Linear Solver}

% TODO: Describe the non-linear solution approach (Newton-Raphson, etc.)

\section{Implementation}

\subsection{Software and Tools}

This work utilizes FEniCS \citep{fenics} for finite element computations, leveraging its powerful variational formulation capabilities.

% TODO: Describe your implementation details

\section{Numerical Experiments}

\subsection{Test Cases}

% TODO: Describe your test cases

\subsection{Validation}

% TODO: Present validation results

\section{Results and Discussion}

\subsection{Velocity Profiles}

% TODO: Present and discuss velocity profiles

\subsection{Sensitivity Analysis}

% TODO: Analyze parameter sensitivity

\section{Conclusion}

% TODO: Summarize findings and future work

\section*{Acknowledgments}

% TODO: Add acknowledgments if needed

\bibliographystyle{plainnat}
\bibliography{bibliography}

\end{document}
